\documentclass{beamer}
% \usetheme{Boadilla}

\title{Version control systems (VCS)}
\author{Joe Steeve}
\date{\today}

\begin{document}
\begin{frame}
  \titlepage
\end{frame}

\begin{frame}
  \frametitle{What is a VCS}
  \begin{enumerate}
  \item A tool to maintain change history to source code over a period
    of time.
  \item Provides mechanisms for teams to collaborate on a single
    project.
  \item Enables developers to branch out and merge back.
  \item Enables developers to rollback to old versions.
  \item Enables auditability and traceability.
  \end{enumerate}
\end{frame}

\begin{frame}
  \frametitle{Types of VCS}

  \begin{description}
  \item[Centralised VCS] Has a central server which serialises the
    commit history.
  \item[Distributed VCS] Decentralised. Commit history is maintained by
    pushing/pulling changesets from different repositories.
  \end{description}
\end{frame}


\begin{frame}
  \frametitle{Centralised VCS}

  \begin{enumerate}
  \item Examples: RCS, CVS, Subversion.
  \item Most operations require access to the central server.
  \item Too much central control.
  \item Some VCS partially support decentralisation.
  \end{enumerate}
\end{frame}

\begin{frame}
  \frametitle{Distributed VCS}
  \begin{enumerate}
  \item Examples: git, mercurial, bazaar, monotone, Darcs
  \item Every developer gets a copy of the entire repository.
  \item Commits are done locally.
  \item Changes are pushed between repositories.
  \item Flexible enough for community development
  \end{enumerate}
\end{frame}

\begin{frame}
  \frametitle{git}
  \begin{enumerate}
  \item Git was created by Linus Torvalds for maintaining the Linux
    kernel.
  \item ``Git'' is a slang for ``Unpleasant person'' in British
    English. But really, there is no meaning.
  \item Close to 90\% adoption
  \end{enumerate}
\end{frame}

\begin{frame}
  \frametitle{Git, Concepts (1)}
  \begin{enumerate}
  \item Clone: Copying a remote repository locally.
  \item Staging: Adding the changes to the index.
  \item Commit: Commit staged changes to the local repository.
  \item Push/Pull: Push/Pull changesets to/from a remote repository.
  \item Branch: A diverged line of development with a history of its
    own, for a particular topic.
  \item Merge: After a topic is finished, it can be merged to the main
    branch.
  \item Tag: A named reference to a particular commit. Can be
    cryptographically signed.
  \item Pull request: A request to peers to review the changes in a
    topic branch and merge it into the main branch.
  \end{enumerate}
\end{frame}

\begin{frame}
  -EoF-
\end{frame}

\end{document}

